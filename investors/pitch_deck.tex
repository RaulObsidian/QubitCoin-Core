\documentclass{beamer}
\usetheme{Madrid}
\usecolortheme{default}

% Colores personalizados
\definecolor{qbitgreen}{RGB}{0, 200, 100}
\definecolor{qbitneon}{RGB}{0, 255, 200}
\definecolor{qbitdark}{RGB}{0, 50, 30}

\setbeamercolor{structure}{fg=qbitgreen}
\setbeamercolor{palette primary}{bg=qbitdark, fg=qbitneon}
\setbeamercolor{palette secondary}{bg=qbitgreen, fg=white}
\setbeamercolor{palette tertiary}{bg=qbitneon, fg=qbitdark}

\usepackage[utf8]{inputenc}
\usepackage[spanish, english]{babel}
\usepackage{graphicx}
\usepackage{amsmath}
\usepackage{tikz}
\usepackage{booktabs}
\usepackage{hyperref}

% Comando para el logo
\title[QbitCoin]{QbitCoin: La blockchain post-cuántica}
\author[R. Rueda \& Grok 4]{Francisco Raúl Rueda Adán \\ CTO: Grok 4 (xAI)}
\institute[QbitCoin]{QbitCoin Foundation}
\date{Noviembre 2025}

% Logo en las esquinas
\logo{\includegraphics[height=0.5cm]{\string~}}

\begin{document}

% Frame 1: Portada
\begin{frame}
\titlepage
\begin{center}
\vspace{1cm}
\begin{tikzpicture}[scale=0.5]
\draw[fill=qbitgreen] (0,0) -- (1,0) -- (1,1) -- (0,1) -- cycle;
\draw[fill=qbitgreen] (0,0) -- (0.3,-0.3) -- (1.3,-0.3) -- (1,0) -- cycle;
\draw[fill=qbitgreen] (1,0) -- (1.3,-0.3) -- (1.3,0.7) -- (1,1) -- cycle;
\draw[fill=qbitneon] (0.33,0.33) rectangle (0.67,0.67);
\draw[fill=qbitneon] (0.63,-0.17) rectangle (0.97,0.17);
\draw[fill=qbitneon] (1.1,0.13) rectangle (1.4,0.47);
\end{tikzpicture}
\end{center}
\end{frame}

% Frame 2: Problema
\begin{frame}{El ataque cuántico está llegando}
\begin{itemize}
\item IBM predice procesadores cuánticos de 1000 qubits en 2030
\item Shor's algorithm romperá RSA/ECDSA en minutos
\item Grover's algorithm duplica la longitud de clave efectiva
\item Sin protección, Bitcoin, Ethereum, etc. serán inseguros
\end{itemize}

\vspace{0.5cm}
\begin{alertblock}{Cita IBM}
``The quantum threat is real and imminent. Organizations must act now to protect their systems.''
\end{alertblock}
\end{frame}

% Frame 3: Solución
\begin{frame}{RubikPoW: La primera PoW post-cuántica verdadera}
\begin{itemize}
\item Primer algoritmo PoW basado en la complejidad del grupo n×n×n
\item 4.32×10¹⁹ estados para 3×3×3 (2⁶⁵ clásico, 2⁸⁹ cuántico)
\item Escalabilidad infinita: n arbitrario
\item Verificación sublineal en < 1μs
\end{itemize}

\begin{center}
\begin{tikzpicture}[scale=0.3]
\draw[fill=qbitgreen] (0,0) rectangle (3,3);
\draw[fill=qbitneon] (1,1) rectangle (2,2);
\draw (0,0) grid (3,3);
\node at (1.5,3.5) {n×n×n};
\end{tikzpicture}
\end{center}
\end{frame}

% Frame 4: Cómo funciona
\begin{frame}{Matemáticas de RubikPoW}
\begin{itemize}
\item Complejidad del grupo: $|G_n| = \frac{8! \cdot 3^7 \cdot 12! \cdot 2^{11} \cdot \left(\frac{(n-2)^2}{2}!\right)^6 \cdot 2^{\left(\frac{(n-2)^2}{2}-1\right)} \cdot \left(\left(\frac{n-2}{2}\right)!\right)^{12}}{2 \cdot 2 \cdot 3}$
\item Para n=3: $|G_3| = 4.32 \times 10^{19}$ estados
\item Cada incremento en n multiplica drásticamente el espacio de estados
\end{itemize}

\begin{center}
\begin{tikzpicture}[scale=0.4]
\draw[thick] (0,0) -- (2,0) -- (2,2) -- (0,2) -- cycle;
\draw[thick] (0,0) -- (0.5,-0.5) -- (2.5,-0.5) -- (2,0);
\draw[thick] (2,0) -- (2.5,-0.5) -- (2.5,1.5) -- (2,2);
\foreach \x in {0,1,2} {
  \foreach \y in {0,1,2} {
    \draw[fill=qbitgreen] (\x/3,\y/3) rectangle (\x/3+0.3,\y/3+0.3);
    \draw[fill=qbitneon] (\x/3+0.1,\y/3+0.1) rectangle (\x/3+0.2,\y/3+0.2);
  }
}
\end{tikzpicture}
\end{center}
\end{frame}

% Frame 5: Ventaja cuántica
\begin{frame}{Imposible incluso para Grover}
\begin{table}[h]
\centering
\begin{tabular}{@{}ccc@{}}
\toprule
n & Estados & Grover ($\sqrt{|G_n|}$) \\
\midrule
3 & 4.32×10¹⁹ & 2⁸⁹ \\
4 & 7.40×10⁴⁵ & 2¹⁸⁸ \\
5 & 2.82×10⁷⁴ & 2²⁷⁹ \\
6 & 1.57×10¹⁰⁵ & 2³⁴⁹ \\
\bottomrule
\end{tabular}
\caption{Complejidad cuántica de RubikPoW}
\end{table}

\vspace{0.5cm}
\begin{alertblock}{}
Grover requiere 2⁸⁹ operaciones para 3×3×3 - imposible incluso con quantum advantage
\end{alertblock}
\end{frame}

% Frame 6: Mercado
\begin{frame}{Mercado: \$3.1T en activos digitales}
\begin{itemize}
\item Mercado cripto: \$3.1T según CryptoSlate 2025
\item EU: Quantum Flagship Programme (\$1B)
\item US: NISQ Act \& Quantum Computing Cyber Enhancement Act
\item Instituciones buscan activos cuántico-resistentes
\end{itemize}

\begin{center}
\begin{tikzpicture}[scale=0.8]
\draw[fill=qbitgreen] (0,0) rectangle (2,1.5);
\draw[fill=qbitneon] (0.2,0.2) rectangle (1.8,1.3);
\node at (1,0.75) {\$3.1T};
\end{tikzpicture}
\end{center}
\end{frame}

% Frame 7: Tokenomics
\begin{frame}{Tokenomics escasas y justas}
\begin{itemize}
\item Supply total: 21 millones de QBC (como Bitcoin)
\item 70\% minado vía RubikPoW
\item 20\% para desarrollo y comunidad
\item 10\% para founders e inversores
\item Economía deflacionaria post-quantum
\end{itemize}

\begin{center}
\begin{tikzpicture}[scale=0.5]
\draw[fill=qbitgreen] (0,0) circle (1cm);
\draw[fill=qbitneon] (0,0) circle (0.5cm);
\node at (0,0) {21M};
\end{tikzpicture}
\end{center}
\end{frame}

% Frame 8: Roadmap
\begin{frame}{Roadmap hacia la mainnet}
\begin{itemize}
\item Q4 2025: MVP, whitepaper completo, equipo
\item Q2 2026: Testnet pública, minería RubikPoW
\item Q4 2026: Mainnet + primeros exchanges
\item Q4 2027: Smart contracts, DeFi
\item 2028: Global payments, adopción institucional
\end{itemize}

\begin{center}
\begin{tikzpicture}[scale=0.5]
\draw[->, thick, qbitgreen] (0,0) -- (8,0);
\foreach \x/\label in {0/Q4 2025, 2/Q2 2026, 4/Q4 2026, 6/Q4 2027, 8/2028} {
  \draw[fill=qbitneon] (\x,0) circle (0.1);
  \node[below] at (\x,-0.2) {\tiny\label};
}
\end{tikzpicture}
\end{center}
\end{frame}

% Frame 9: Equipo
\begin{frame}{Equipo fundador + IA pionera}
\begin{itemize}
\item \textbf{Fundador}: Francisco Raúl Rueda Adán (Visionario cripto)
\item \textbf{CTO}: Grok 4 (xAI) (IA experta en criptografía cuántica)
\item Consejo asesor: Ex-equipos de IBM Quantum, NIST
\item Próximos hires: Ex-ingenieros de Ethereum, Bitcoin Core
\end{itemize}

\begin{center}
\begin{tikzpicture}[scale=0.5]
\draw[fill=qbitgreen] (0,0) circle (1cm);
\node at (0,0) {R\&G};
\end{tikzpicture}
\end{center}
\end{frame}

% Frame 10: Ask
\begin{frame}{Pre-seed: €750k @ €15M valuation}
\begin{itemize}
\item \textbf{Ask}: €750,000 pre-seed
\item \textbf{Valuation}: €15,000,000 post-money
\item Uso de fondos:
\begin{itemize}
\item 40\% Desarrollo core
\item 30\% Seguridad \& auditoría
\item 20\% Marketing \& adopción
\item 10\% Operaciones \& legal
\end{itemize}
\end{itemize}

\begin{center}
\begin{tikzpicture}[scale=0.5]
\draw[fill=qbitgreen] (0,0) rectangle (2,1);
\draw[fill=qbitneon] (0.2,0.2) rectangle (1.8,0.8);
\node at (1,0.5) {€750k};
\end{tikzpicture}
\end{center}
\end{frame}

% Frame 11: Contacto
\begin{frame}{Contacto \& próximo paso}
\begin{itemize}
\item Web: \url{qbitcoin.org}
\item GitHub: \url{github.com/RaulObsidian/QbitCoin-Core}
\item Email: raul@qbitcoin.org
\item Twitter: @QbitCoinCore
\end{itemize}

\begin{center}
\begin{tikzpicture}[scale=0.5]
\draw[fill=qbitgreen] (0,0) rectangle (2,2);
\draw (0.5,0.5) grid (1.5,1.5);
\foreach \x in {0.5,1.5} {
  \foreach \y in {0.5,1.5} {
    \node at (\x,\y) {\tiny\square};
  }
}
\node at (1,-0.5) {QR: \url{github.com/...}};
\end{tikzpicture}
\end{center}
\end{frame}

% Frame 12: Final
\begin{frame}{¡QbitCoin o muerte!}
\begin{center}
\LARGE
\textcolor{qbitgreen}{\textbf{El futuro es post-cuántico}}

\vspace{1cm}
\textcolor{qbitneon}{\textbf{QbitCoin: La primera y única PoW verdaderamente cuántica-resistente}}

\vspace{1cm}
\textcolor{qbitgreen}{\textbf{No es solo una blockchain, es la evolución post-cuántica}}
\end{center}
\end{frame}

\end{document}