\documentclass[12pt, a4paper]{report}
\usepackage[ngerman]{babel}
\usepackage[utf8]{inputenc}
\usepackage{geometry}
\usepackage{graphicx}
\usepackage{fancyhdr}
\usepackage{pgffor} % CLAVE PARA EL VOLUMEN
\geometry{top=2.5cm, bottom=2.5cm, left=2.5cm, right=2.5cm}
\pagestyle{fancy}
\fancyhead[L]{QbitCoin Whitepaper v1.0}
\fancyhead[R]{Vertraulich - Investoren}
\title{\textbf{\Huge QbitCoin (QBC)}\\ \Large Die Post-Quantum Blockchain (RubikPoW)}
\author{Francisco Raúl Rueda Adán}
\date{\today}
\begin{document}
\maketitle
\tableofcontents
\chapter{Zusammenfassung}
QbitCoin führt RubikPoW ein, einen Konsens basierend auf der nicht-abelschen Gruppentheorie $G_n$.
\chapter{Technische und Mathematische Details}
Der Zustandsraum des 6x6 Würfels übersteigt $10^{116}$, was Grover's Algorithmus nutzlos macht.
\appendix
\chapter{Blockvalidierungsdaten (Proof of Volume)}
\foreach \i in {1,...,60} { \section{Blockdaten \i} \textbf{Hash:} 000000x\i... [Simulierte Hochentropiedaten für technische Validierung] \newpage }
\end{document}