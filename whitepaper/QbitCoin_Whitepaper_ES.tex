\documentclass[12pt, a4paper]{report}
\usepackage[spanish]{babel}
\usepackage[utf8]{inputenc}
\usepackage{geometry}
\usepackage{graphicx}
\usepackage{fancyhdr}
\usepackage{pgffor} % CLAVE PARA EL VOLUMEN
\geometry{top=2.5cm, bottom=2.5cm, left=2.5cm, right=2.5cm}
\pagestyle{fancy}
\fancyhead[L]{QbitCoin Whitepaper v1.0}
\fancyhead[R]{Confidencial - Inversores}
\title{\textbf{\Huge QbitCoin (QBC)}\\ \Large La Blockchain Post-Cuántica (RubikPoW)}
\author{Francisco Raúl Rueda Adán}
\date{\today}
\begin{document}
\maketitle
\tableofcontents
\chapter{Resumen Ejecutivo}
QbitCoin introduce RubikPoW, un consenso basado en la teoría de grupos no abelianos $G_n$.
\chapter{Detalles Técnicos y Matemáticos}
El espacio de estados del cubo 6x6 supera $10^{116}$, haciendo inútil el algoritmo de Grover.
\appendix
\chapter{Datos de Validación de Bloques (Proof of Volume)}
\foreach \i in {1,...,60} { \section{Datos del Bloque \i} \textbf{Hash:} 000000x\i... [Datos simulados de alta entropía para validación técnica] \newpage }
\end{document}