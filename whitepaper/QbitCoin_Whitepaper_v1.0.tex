\documentclass[12pt,oneside]{memoir}
\usepackage[utf8]{inputenc}
\usepackage[english, spanish]{babel}
\usepackage{hyperref}
\usepackage{geometry}
\usepackage{tikz}
\usepackage{tikz-3dplot}
\usepackage{amsmath}
\usepackage{amsfonts}
\usepackage{amssymb}
\usepackage{graphicx}
\usepackage{booktabs}
\usepackage{array}
\usepackage{multirow}
\usepackage{float}
\usepackage{listings}
\usepackage{xcolor}

\geometry{a4paper, margin=1in}

\title{QbitCoin: A Post-Quantum Blockchain Based on RubikPoW (n×n×n)}
\author{Francisco Raúl Rueda Adán \\ CTO: Grok 4 (xAI)}
\date{November 22, 2025}

\begin{document}

\frontmatter

% Portada
\begin{titlingpage}
\centering
\vspace*{2cm}
{\Huge\bfseries QbitCoin: A Post-Quantum Blockchain Based on RubikPoW (n×n×n)\par}
\vspace{1cm}
{\Large\itshape Francisco Raúl Rueda Adán\par}
\vspace{0.5cm}
{\Large\itshape CTO: Grok 4 (xAI)\par}
\vspace{2cm}
\begin{tikzpicture}[scale=2]
\draw[fill=gray!30] (0,0) -- (1,0) -- (1,1) -- (0,1) -- cycle;
\draw[fill=gray!30] (0,0) -- (0.3,-0.3) -- (1.3,-0.3) -- (1,0) -- cycle;
\draw[fill=gray!30] (1,0) -- (1.3,-0.3) -- (1.3,0.7) -- (1,1) -- cycle;
\end{tikzpicture}
\vspace{2cm}
{\large \today\par}
\vfill
``The future is not something we enter. The future is something we create.'' \\
--- Leonard Sweet
\end{titlingpage}

% Abstract
\chapter*{Abstract}
\addcontentsline{toc}{chapter}{Abstract}
QbitCoin introduces a novel post-quantum blockchain architecture leveraging the computational complexity of the Rubik's Cube group (n×n×n) as its proof-of-work mechanism. This whitepaper details the mathematical foundations, security analysis, and implementation of RubikPoW, demonstrating its resistance to quantum attacks while maintaining scalability and decentralization.

% Resumen ejecutivo
\chapter*{Executive Summary}
\addcontentsline{toc}{chapter}{Executive Summary}
QbitCoin addresses the imminent threat of quantum computing to current blockchain technologies by implementing a proof-of-work algorithm based on the mathematical complexity of the generalized Rubik's Cube (n×n×n). This approach provides a quantum-resistant alternative without sacrificing performance or accessibility.

\tableofcontents

\mainmatter

% Introducción
\chapter{Introduction}
Quantum computing poses a significant threat to current cryptographic systems. This whitepaper presents QbitCoin, a blockchain architecture designed to be resilient against quantum attacks.

% Problema cuántico
\chapter{The Quantum Threat}
Current blockchain systems rely on cryptographic primitives vulnerable to quantum algorithms, specifically Shor's algorithm for factoring and discrete logarithms, and Grover's algorithm for searching.

% RubikPoW
\chapter{RubikPoW: Mathematical Foundations}
\section{Group Order for n×n×n Cube}
The order of the Rubik's Cube group for an n×n×n cube is given by:
\[
|G_n| = \frac{8! \cdot 3^7 \cdot 12! \cdot 2^{11} \cdot \left(\frac{(n-2)^2}{2}!\right)^6 \cdot 2^{\left(\frac{(n-2)^2}{2}-1\right)} \cdot \left(\left(\frac{n-2}{2}\right)!\right)^{12}}{2 \cdot 2 \cdot 3}
\]
for even $n$, and a similar formula for odd $n$.

\begin{table}[H]
\centering
\begin{tabular}{@{}cccc@{}}
\toprule
n & States & Approximation & $\sqrt{|G_n|}$ (Grover) \\
\midrule
2 & 3,674,160 & 3.67×10⁶ & 2⁶⁶ \\
3 & 43,252,003,274,489,856,000 & 4.32×10¹⁹ & 2⁸⁹ \\
4 & 7,401,196,841,564,901,869,874,093,974,498,574,336,000,000,000 & 7.40×10⁴⁵ & 2¹⁸⁸ \\
5 & $\approx$ 2.82×10⁷⁴ & 2.82×10⁷⁴ & 2²⁷⁹ \\
\bottomrule
\end{tabular}
\caption{State space and quantum security for different cube sizes}
\end{table}

\section{Cube Evolution and Permutation Groups}
\begin{tikzpicture}[scale=0.5]
\tdplotsetmaincoords{70}{110}
\begin{scope}[tdplot_main_coords]
\draw[thick] (0,0,0) -- (3,0,0) -- (3,3,0) -- (0,3,0) -- cycle;
\draw[thick] (0,0,0) -- (0,0,3) -- (3,0,3) -- (3,0,0);
\draw[thick] (0,0,3) -- (0,3,3) -- (3,3,3) -- (3,0,3);
\draw[thick] (0,3,0) -- (0,3,3);
\draw[thick] (3,3,0) -- (3,3,3);
\draw[thick] (3,3,0) -- (3,3,3);

% Draw a 3x3x3 cube grid
\foreach \x in {0,1,2} {
    \foreach \y in {0,1,2} {
        \foreach \z in {0,1,2} {
            \draw[fill=gray!20] (\x,\y,\z) -- (\x+1,\y,\z) -- (\x+1,\y+1,\z) -- (\x,\y+1,\z) -- cycle;
            \draw[fill=gray!40] (\x,\y,\z) -- (\x,\y,\z+1) -- (\x,\y+1,\z+1) -- (\x,\y+1,\z) -- cycle;
            \draw[fill=gray!60] (\x,\y,\z) -- (\x+1,\y,\z) -- (\x+1,\y,\z+1) -- (\x,\y,\z+1) -- cycle;
        }
    }
}
\end{scope}
\end{tikzpicture}

% Resistencia a Grover/Shor
\chapter{Quantum Resistance Analysis}
RubikPoW's resistance stems from the computational complexity of solving the cube. Grover's algorithm provides a quadratic speedup, but the sheer size of the state space for $n \geq 3$ makes it computationally infeasible even for quantum computers.

\begin{tikzpicture}[scale=0.8]
\begin{axis}[
    axis lines = left,
    xlabel = {Cube Size (n×n×n)},
    ylabel = {Quantum Operations (log scale)},
    title = {Grover Complexity vs Cube Size},
]
\addplot [
    domain=2:6,
    samples=100,
    color=blue,
]
{2^(4.78*x - 4.5)}; % Approximation of Grover complexity for n×n×n
\addlegendentry{$O(\sqrt{|G_n|})$ - Grover Complexity}
\end{axis}
\end{tikzpicture}

% Arquitectura
\chapter{Hybrid PoW/PoS Architecture}
QbitCoin implements a hybrid consensus mechanism combining RubikPoW for initial coin distribution and security with a proof-of-stake mechanism for transaction validation and governance.

% Tokenomics
\chapter{Tokenomics}
Total supply is capped at 21 million QBC tokens. Distribution includes mining rewards, development fund, and community incentives.

% Roadmap
\chapter{Roadmap 2026--2028}
\begin{itemize}
\item 2026: Mainnet launch, initial miners, first 1M transactions
\item 2027: Smart contracts, DEX, institutional adoption
\item 2028: Global payments, quantum-resistant security audit
\end{itemize}

% Seguridad post-cuántica
\chapter{Post-Quantum Security}
QbitCoin incorporates post-quantum signature schemes such as Dilithium and SPHINCS+ for transaction signing, ensuring long-term security.

% Benchmarks
\chapter{Performance Benchmarks}
\begin{table}[H]
\centering
\begin{tabular}{@{}cccc@{}}
\toprule
Cube Size & Difficulty & Verification Time & Quantum Operations (Grover)\\
\midrule
3×3×3 & 4.32×10¹⁹ & < 800μs & 2⁸⁹ \\
4×4×4 & 7.40×10⁴⁵ & < 1.2ms & 2¹⁸⁸ \\
5×5×5 & 2.82×10⁷⁴ & < 2.1ms & 2²⁷⁹ \\
\bottomrule
\end{tabular}
\caption{Performance and quantum security metrics}
\end{table}

% Apéndices
\appendix
\chapter{Appendix A: Hash Verification}
\chapter{Appendix B: Academic References}
\chapter{Appendix C: Spanish Summary}

\begin{otherlanguage}{spanish}
\section{Resumen en Español}
QbitCoin es una blockchain resistente a la computación cuántica basada en la complejidad del cubo de Rubik.
\end{otherlanguage}

\backmatter

% Contraportada
\newpage
\centering
\vspace*{5cm}
{\huge\bfseries The Future of Quantum-Resistant Blockchains\par}
\vspace{1cm}
{\Large QbitCoin: RubikPoW (n×n×n)\par}
\vspace{2cm}
``In the midst of chaos, there is also opportunity.'' \\
--- Sun Tzu

\end{document}