\documentclass[11pt, a4paper]{report}
\usepackage[spanish]{babel}
\usepackage[utf8]{inputenc}
\usepackage{geometry}
\usepackage{amsmath, amssymb, amsfonts}
\usepackage{graphicx}
\usepackage{fancyhdr}
\usepackage{tikz}
\usepackage{booktabs}
\usepackage{hyperref}
\usepackage{titlesec}
\usepackage{xcolor}

% Configuración de diseño profesional
\geometry{top=3cm, bottom=3cm, left=2.5cm, right=2.5cm}
\pagestyle{fancy}
\fancyhead[L]{\textbf{QbitCoin Core}}
\fancyhead[R]{Protocolo de Seguridad Post-Cuántica}
\definecolor{qbitblue}{RGB}{0, 51, 102}

\title{
    \vspace{2cm}
    \textbf{\Huge \color{qbitblue} QbitCoin (QBC)} \\
    \vspace{1cm}
    \Large La Primera Blockchain con Consenso basado en \\ Grupos de Permutación No Abelianos (RubikPoW) \\
    \vspace{2cm}
    \textbf{WHITEPAPER TÉCNICO v2.0} \\
    \large Edición Institucional
}
\author{\textbf{Francisco Raúl Rueda Adán} \\ Fundador y Arquitecto Jefe}
\date{Diciembre 2025}

\begin{document}

\maketitle

\begin{abstract}
\textbf{Resumen Ejecutivo:} La inminente llegada de la computación cuántica (Q-Day) dejará obsoletos los algoritmos RSA y de Curva Elíptica (ECDSA) que protegen actualmente el 99\% de la economía digital mundial, incluido Bitcoin. QbitCoin introduce una solución definitiva: \textbf{RubikPoW}. Un mecanismo de prueba de trabajo basado en la complejidad combinatoria de resolver espacios de estado en subgrupos $G_n$, resistente matemáticamente al Algoritmo de Shor y Grover. Este protocolo implementa una arquitectura de seguridad escalonada (del 3K al 6K) para ofrecer utilidad real desde micropagos hasta secretos de estado.
\end{abstract}

\tableofcontents
\listoftables
\newpage

\chapter{La Amenaza Cuántica y la Obsolescencia de Bitcoin}
\section{El Algoritmo de Shor}
Los ordenadores cuánticos utilizan cúbits y superposición para resolver la factorización de enteros en tiempo polinómico $O((\log N)^3)$. Esto significa que las claves privadas de Bitcoin (secp256k1) podrán ser derivadas de las claves públicas en cuestión de horas.

\section{La Solución QbitCoin}
QbitCoin abandona la aritmética modular en favor de la \textbf{Teoría de Grupos}. La seguridad de nuestra red no depende de factorizar números, sino de encontrar el camino más corto ("Número de Dios") en un espacio de permutaciones de alta entropía.

\chapter{Tecnología RubikPoW: El Motor del Consenso}
\section{Fundamentos Matemáticos}
El puzzle criptográfico se basa en el Grupo del Cubo $G$. Para un cubo de dimensiones $N \times N \times N$, el espacio de estados $\Omega$ crece super-exponencialmente.

\begin{equation}
    Size(N) = \frac{8! \cdot 3^7 \cdot 12! \cdot 2^{11}}{2} \approx 4.3 \times 10^{19} \quad (\text{para } N=3)
\end{equation}

Para nuestro nivel máximo ($N=6$), el espacio de búsqueda supera $1.57 \times 10^{116}$, una cifra mayor que el número de átomos en el universo observable.

\chapter{Arquitectura de Seguridad Escalonada (Tiered Security)}
A diferencia de las blockchains monolíticas, QbitCoin ofrece niveles de encriptación adaptativos según la criticidad de la transacción.

\begin{table}[h]
\centering
\renewcommand{\arraystretch}{1.5}
\begin{tabular}{|c|c|l|l|}
\hline
\textbf{Nivel} & \textbf{Estructura} & \textbf{Usuario Objetivo} & \textbf{Aplicación Real} \\
\hline
\textbf{3K} & Cubo 3x3x3 & Usuario Estándar & Pagos diarios, compras online. \\
\hline
\textbf{4K} & Cubo 4x4x4 & Corporativo & Contratos B2B, Nóminas. \\
\hline
\textbf{5K} & Cubo 5x5x5 & Institucional & Banca, Reservas Federales. \\
\hline
\textbf{6K} & Cubo 6x6x6 & Militar/Científico & Secretos de Estado, Datos Genéticos. \\
\hline
\end{tabular}
\caption{Matriz de Seguridad y Casos de Uso de QbitCoin}
\end{table}

\section{Escalabilidad Infinita (XK-XK-XK)}
El protocolo está diseñado para ser agnóstico a la dimensión. A medida que la computación cuántica avance, la red puede activar mediante soft-fork niveles superiores (7K, 8K...), garantizando la seguridad perpetua.

\chapter{Tokenomics y Economía del Protocolo}
Diseñado para ser dinero duro, escaso y deflacionario.
\begin{itemize}
    \item \textbf{Suministro Total:} 21,000,000 QBC (Inmutable).
    \item \textbf{Halving:} Cada 210,000 bloques.
    \item \textbf{Recompensa Minera:} Basada en la dificultad del cubo resuelto. Resolver un bloque 6K otorga mayores recompensas que un bloque 3K, incentivando la inversión en hardware de alta computación.
\end{itemize}

\chapter{Conclusión}
QbitCoin no es solo una criptomoneda, es la infraestructura de seguridad para la era post-cuántica. Mientras otras redes tendrán que migrar o morir, QbitCoin ha nacido preparada.

\end{document}